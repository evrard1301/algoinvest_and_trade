\begin{frame}{Une première solution}

  \begin{block}{Une solution \textit{Bruteforce}}
    La solution la plus simple est d'énumérer toutes les possibilités
    puis de choisir la meilleure.
  \end{block}

  \begin{block}{Principe}
    Générer \textbf{tout les investissements possibles} puis les trier
    et enfin choisir le meilleur.
  \end{block}

  \begin{alertblock}{Attention}
    Il faut générer les combinaisons et non pas les permutations.
    Par exemple, $^{5}C_3 = 10$ et $^{5}P_3 = 60$.
  \end{alertblock}

\end{frame}


\begin{frame}[fragile]{Algorithme de la solution naïve}
  \begin{block}{Pseudo-Code}
    \tiny
    \begin{center}
      \begin{lstlisting}
        BRUTEFORCE(actions, current, cost, profit, solutions, index)

          si cost <= 500
            solutions = solutions (*$\cup$*) {(current, cost, profit)}
          fin si

          pour i de index a |actions|
            bruteforce(actions \ actions[i],
              current (*$\cup$*) {action},
              cost + C(action),
              profit + P(action),
              solutions,
              i)
          fin pour

          retourner solutions
        FIN
      \end{lstlisting}

      Pour trouver le meilleur investissement:

      \begin{lstlisting}
        MEILLEUR-CHOIX(actions)
          solutions = BRUTEFORCE(actions, (*$\emptyset$*), 0, 0, (*$\emptyset$*), 0)
          TRI(solutions)
          retourner DERNIER-ELEMENT(solutions)
        FIN
      \end{lstlisting}
    \end{center}
  \end{block}
\end{frame}

\begin{frame}{Complexité asymptotique en temps de la solution naïve}

  \begin{block}{Forme générale}
    L'algorithme \textit{bruteforce} (BF) est récursif:
    \begin{equation}
      BF(n) = |actions| \times BF(n - 1) + \mathcal{O}(n)
    \end{equation}
  \end{block}

  \begin{block}{Soit $TOTAL$ le nombre d'appels récursifs de \textit{bruteforce}}
    $TOTAL = |actions| \times (|actions| - 1) \times (|actions| - 2) \times \cdots \times 1 = |actions|!$
  \end{block}

  \begin{block}{Par conséquent, nous avons:}
    \begin{equation}
      BRUTEFORCE = \mathcal{O}(n!)
    \end{equation}
  \end{block}

\end{frame}

\begin{frame}{Complexité asymptotique: une seconde approche}
  \begin{block}{}
    Nous énumerons toutes les combinaisons d'actions possibles.  Pour
    une entrée de $n = 20$ actions, nous avons $\binom{n}{k}$
    solutions avec $k$ la taille de la sortie.
  \end{block}

  \begin{block}{}
    \begin{equation}
      \binom{n}{k} = \frac{n!}{(n - k)! \times k!}
    \end{equation}

  \end{block}
\end{frame}

\begin{frame}[fragile]{Complexité asymptotique en mémoire de la solution naïve}
  \begin{block}{Dans le pire cas:}
    On ajoute au tableau de solutions une action à chaque itérations.
    \begin{center}
      \tiny
      \begin{lstlisting}[numbers=none, frame=none]
          si cost <= 500
            solutions = solutions (*$\cup$*) {(*$\cdots$*)}
          fin si
      \end{lstlisting}
    \end{center}
  \end{block}

  \begin{block}{Taille finale du tableau}
    $|actions| \times (|actions| - k)\times \cdots \times 1 = |actions|!$.
  \end{block}

  \begin{block}{Conclusion:}
    \begin{equation}
      BRUTEFORCE = \mathcal{O}(n!)
    \end{equation}
  \end{block}
\end{frame}

\begin{frame}{Conclusion de la solution naïve}
  \begin{block}{Problèmes}
    \begin{itemize}
    \item $\mathcal{O}(n!)$ grandit bien trop vite.
    \item On génère tout les cas, ce qui est efficace mais inefficient.
    \end{itemize}
  \end{block}

  \begin{block}{Conclusion}
    $\rightarrow$ L'algorithme BRUTEFORCE n'est pas utilisable dans le
    monde réel.
  \end{block}
\end{frame}
